\chapter{Introdução}

A crescente evolução dos computadores e sua capacidade de processamento vem influenciando diretamente as mais diversas áreas do conhecimento. A Biologia
tem aproveitado o crescimento do poder computacional para realizar análises cada vez mais poderosas sobre espécies, organismos e sua carga genética.
A possibilidade de comparação de grandes sequências de DNA em um curto intervalo de tempo possibilita resultados impossíveis de serem obtidos através
de uma comparação manual.

\newglossaryentry{engsoft}{name=Engenharia de Software,
description={\hspace{2 mm}área do conhecimento da computação voltada para a especificação, desenvolvimento e manutenção de sistemas de software
aplicando tecnologias e práticas de gerência de projetos e outras disciplinas, objetivando organização, produtividade e qualidade}}

\newglossaryentry{bioinformatica}{name=Bioinformática,
description={\hspace{2 mm}corresponde a aplicação das técnicas da Informática, no sentido de análise da informação na área de estudo da Biologia, utilizando
tais técnicas computacionais e matemáticas à geração e gerenciamento de bioinformação}}

A \gls{bioinformatica} é hoje uma das áreas multidisciplinares mais promissoras, e a comparação de sequências proteicas permite um impulso na área de Análise
Filogenética, onde diversos métodos tem surgido. Outras áreas do conhecimento que têm se beneficiado bastante com a crescente evolução computacional são a
Física e a Matemática que, assim como a Biologia, podem atingir resultados rapidamente antes impossíveis, e acabam gerando um leque de opções para pesquisa
em computação aplicada.

\newglossaryentry{fisica-estatistica}{name=Física Estatística,
description={\hspace{2 mm}é o ramo da Física que usa métodos da teoria das probabilidades e estatística e, particularmente, as ferramentas matemáticas para
lidar com grandes populações e aproximações, na solução de problemas físicos. Seu principal objetivo é esclarecer as propriedades da matéria sob conjuntos,
em termos de leis físicas que regem o movimento atômico}}

\newglossaryentry{sistemas-complexos}{name=Sistemas Complexos,
description={\hspace{2 mm}são sistemas onde suas propriedades não são uma consequência natural de seus elementos constituintes vistos isoladamente. As
propriedades emergentes de um sistema complexo decorrem em grande parte da relação não-linear entre as partes}}

Hoje em dia a demanda por profissionais e pesquisadores de Ciência da Computação nestas áreas tem se mostrado muito grande, e a junção de duas ou mais delas
em um contexto multidisciplinar tem guiado as pesquisas a resultados muito interessantes, além de gerar produtos e também ideias para novos projetos de
pesquisa. Em um contexto de \gls{fisica-estatistica} e \gls{sistemas-complexos}, aliando-se conceitos matemáticos da Teoria dos Grafos, foi possível o
desenvolvimento de um método para Análise Filogenética, que levou a resultados comparáveis com os de outros métodos existentes \cite{andrade2011}.


\newglossaryentry{artefatos}{name=artefatos computacionais,
description={\hspace{2 mm}são vários tipos de subprodutos concretos produzido durante o desenvolvimento de software. Alguns artefatos (por exemplo,
casos de uso, diagramas de classes e outros modelos UML, requisitos e documentos de projeto) ajudam a descrever a função, arquitetura e o design do software.
Outros artefatos estão relacionados com o próprio processo de desenvolvimento - tais como planos de projetos, processos de negócios e avaliações de risco.
Podem ser manuais, arquivos executáveis, módulos etc}}


\newglossaryentry{persistencia}{name=persistência,
description={\hspace{2 mm}refere-se ao armazenamento não-volátil de dados, por exemplo, o armazenamento em um dispositivo físico como um disco rígido.
Pode-se dizer que de maneira geral, o termo persistência é associado a uma ação que consiste em manter em meio físico recuperável, como banco de dados
ou arquivo, de modo a garantir a permanência das informações de um determinado estado de um objeto lógico.}}

Nos três anos de pesquisa e desenvolvimento do método, houve uma maior preocupação com resultados, o que fez com que o processo de desenvolvimento de
\gls{artefatos} para que se chegasse a eles não seguisse um metodologia de desenvolvimento baseada em conceitos de \gls{engsoft}. Como
resultado desse processo de construção de software sem metodologia, o ambiente computacional gerado necessita de
uma curva de aprendizado grande para a sua execução, além de cuidados com relação à organização e nomenclatura de arquivos, configurações,
formatação de dados de entrada e saída para o uso em diferentes programas. A partir do momento em que o método está concluído, testado e comparado com
outros métodos, se faz necessário sua organização para diminuir esta curva de aprendizado. A aplicação de uma metodologia de Engenharia de Software ao
sistema atual se dá a partir de (i) concepção de uma arquitetura que permita uma separação em camadas de interação com o usuário, gerenciamento de
e controle, e \gls{persistencia}; (ii) controle de execução do processo; (iii) possibilidade de extensibilidade; (iv) utilização de código legado.

\section{Motivação}

A proposta deste trabalho surgiu a partir do início de 2007, quando o grupo de pesquisa Física Estatística e Sistemas Complexos (FESC),
do Instituto de Física da Universidade Federal da Bahia, resolveu utilizar sua experiência com Redes Complexas aplicando-a a Análise Filogenética.
Os resultados obtidos com a aplicação do método a dados biológicos abrem espaço para a utilização do mesmo como uma alternativa aos métodos
já existentes no contexto da pesquisa em Análise Filogenética. A partir do momento em que esta alternativa se torna viável, na medida em que traz resultados
válidos e comparáveis aos dos métodos já conhecidos,
faz-se necessária a sua organização para que seja utilizada em larga escala, além de sua extensão,
resultando em um ambiente em que haja a facilidade de uso, desenvolvimento, manutenção e extensão, além da possibilidade de compartilhamento de
resultados entre os diferentes métodos existentes.

\section{Objetivos} \label{sec:objetivos}

O objetivo deste trabalho é aplicar metodologias baseadas em Engenharia de Software de forma a desenvolver um ambiente integrado que agrupe
um conjunto de ferramentas e códigos legados desenvolvidos pelo grupo de pesquisa FESC (Física Estatística e Sistemas Complexos),
utilizando-se de pouca metodologia. Em seguida, um levantamento do que era necessário para que
se pudesse organizar os programas e \textit{scripts} utilizados objetivando melhorar sua execução e experiência de utilização (a serem comprovadas após validação)
 por pesquisadores que detêm de
pouco ou quase nenhum conhecimento de desenvolvimento de software e ambientes Linux, além de permitir sua evolução e extensibilidade.


\newglossaryentry{modelagem}{name=modelagem,
description={\hspace{2 mm}atividade de construir modelos que expliquem as características ou o comportamento de um \textit{software}.
Na construção do \textit{software} os modelos podem ser usados na identificação das características e funcionalidades que o \textit{software} deverá prover
(análise de requisitos), e no planejamento de sua construção}}


A utilização de uma metodologia com a definição de requisitos, \gls{modelagem} e criação de artefatos e implementação do sistema são também
objetivos do trabalho, pois são
essenciais para garantir um bom projeto de \textit{software}, que consequentemente possar ser utilizado, mantido, extendido
e evoluído facilmente.

\sigla{FESC}{Grupo de pesquisa em Física Estatística e Sistemas Complexos - Instituto de Física - UFBA}

\section{Contribuições}

\newglossaryentry{diagrama-casos-uso}{name=Diagrama de Casos de Uso,
description={\hspace{2 mm}descreve a funcionalidade proposta para um novo sistema que será projetado. Pode ser tido como um
documento narrativo que descreve a sequência de eventos de um ator que usa um sistema para completar um processo}}

\newglossaryentry{diagrama-classes}{name=Diagrama de Classes,
description={\hspace{2 mm}é uma representação da estrutura e relações das classes que servem de modelo para objetos. É uma modelagem muito útil para o
sistema, define todas as classes que o sistema necessita possuir e é a base para a construção dos diagramas de comunicação, sequência e estados}}

\newglossaryentry{diagrama-sequencia}{name=Diagrama de Sequências,
description={\hspace{2 mm}descreve a maneira como os grupos de objetos colaboram em algum comportamento ao longo do tempo. Ele registra o comportamento
de um único caso de uso e exibe os objetos e as mensagens passadas entre esses objetos no caso de uso}}

Este trabalho propõe o desenvolvimento de um pacote de \textit{software} com uma preocupação na modelagem bem elaborada, incluindo requisitos e
interações entre o usuário
e o sistema (\gls{diagrama-casos-uso}), organização de dados e informações (\gls{diagrama-classes}) e dinâmica de execução do sistema (\gls{diagrama-sequencia}).
O trabalho de modelagem virá acompanhado de um protótipo que inclui algumas das funcionalidades dispostas nos requisitos.

O sistema beneficiará dois grupos de usuários: (i) desenvolvedores, com um modelo de sistema e uma documentação abrangente, permitindo sua
manutenção, inclusão das funcionalidades levantadas nos requisitos e ainda não implementadas e extensão do modelo para novos requisitos; e (ii) pesquisadores,
com um ambiente integrado que permitirá a utilização e execução de Análise Filogenética através de Redes Complexas por uma interface gráfica,
armazenamento e organização de
arquivos, salvamento e retomada de execuções das análises de forma transparente, direcionado para pesquisadores que não são da área de Ciência da Computação.

\section{Estrutura do trabalho}

Esta monografia está organizada da seguinte maneira:

No capítulo \ref{cap:analisefilo} apresenta-se a fundamentação teórica com seus principais conceitos
e os passos para a execução do método de Análise Filogenética através de Redes Complexas: escolha das sequências proteicas,
similaridade, construção e caracterização de redes, limiar crítico e entremeação, incluindo requisitos e a proposta.

No capítulo \ref{cap:navi}, mostra-se o projeto do sistema propriamente dito, e suas questões
de escopo, organização das informações, casos de uso, modelagem de dados e sua dinâmica de execução através dos diagramas de sequência.

No capítulo \ref{cap:resultados} são mostrados detalhes de implementação, ambiente de execução, ferramentas de
desenvolvimento, além de discutidos quais objetivos foram alcançados, e levantadas outras questões importantes e dificuldades encontradas.

Por fim, é apresentada a conclusão do trabalho e os trabalhos futuros (no capítulo \ref{cap:conclusao}), além de anexos e referências bibliográficas.

% use sua propria estrutura

%Problema etc

%Objetivo etc

%Resultados esperados etc

%Estrutura/Organizao etc
