\chapter{Introdução}

A crescente evolução dos computadores e sua capacidade de processamento vem influenciando diretamente as mais diversas áreas do conhecimento. A Biologia
vem aproveitado o crescimento do poder computacional para realizar análises cada vez mais poderosas sobre espécies, organismos e sua carga genética.
A possibilidade de comparação de grandes sequências de DNA em um curto intervalo de tempo possibilita resultados impossíveis de serem obtidos através
de uma comparação manual.

A Bioinformática é hoje uma das áreas multidisciplinares mais promissoras, e a compração de sequências proteicas permite um impulso na área de Análise
Filogenética, onde diversos métodos tem surgido. Outras áreas do conhecimento que têm se beneficiado bastante com a crescente evolução computacional são a
Física e a Matemática que, assim como a Biologia, podem atingir resultados rapidamente antes impossíveis, e acabam gerando um leque de opções para pesquisa
em computação aplicada.

Hoje em dia a demanda por profissionais e pesquisadores de Ciência da Computação nestas áreas tem se mostrado muito grande, e a junção de duas ou mais destas
áreas em um contexto multidisciplinar tem guiado as pesquisas a resultados espetaculares, além de gerar produtos e também ideias para novos projetos de
pesquisa. Em um contexto de Física Estatística e Sistemas Complexos, aliando-se conceitos matemáticos \cite{andrade2009} e Teoria dos Grafos, foi possível o
desenvolvimento de um método para Análise Filogenética pelo FESC, que obteve bons resultados na comparação com outros métodos existentes \cite{andrade2011}.


Acho que o ponto principal não esta claro. Como seu foco é em engenharia de software, esse é um exemplo de construção de software sem metodologia.
De destaque ao seu trabalho em levar engenharia de software ao sistema atual, controlando o que ja existe e permitindo extensibilidade, sem perder
o codigo legado.


Nos três anos de pesquisa e desenvolvimento do método, houve uma maior atenção voltada para resultados, o que fez com que o processo para que se chegasse
a eles ficasse prejudicado, representando um exemplo de construção de software sem metodologia, e com isso necessitando de uma curva de aprendizado grande
para a sua execução, além de cuidados com relação a organização e nomenclatura de arquivos, configurações, formatação de dados de entrada e saída para o
uso em diferentes programas. A partir do momento em que o método está concluído, testado e comparado com outros métodos, se faz necessário sua organização
para diminuir esta curva de aprendizado, levando Engenharia de Software ao sistema atual a partir de (i) concepção de uma arquitetura que permita uma
separação em camadas de interação com o usuário, lógica de negócio e persistência; (ii) controle de execução do processo; (iii) possibilidade de
extensibilidade; (iv) utilização de código legado.

\section{Motivação}

A proposta deste trabalho surgiu a partir do início de 2007, quando o grupo de Física Estatística e Sistemas Complexos (FESC),
do Instituto de Física da Universidade Federal da Bahia, resolveu utilizar sua experiência com Redes Complexas aplicando-a a Análise Filogenética.
Os resultados obtidos com a aplicação do método a dados biológicos abrem espaço para a utilização do mesmo como uma alternativa aos métodos
já existentes no contexto da pesquisa em Análise Filogenética. A partir do momento em que surge uma alternativa viável a métodos já conhecidos
e bastante aplicados na Biologia com capacidade de mostrar resultados válidos e comparáveis às alternativas existentes, e valendo-se da vantagem
de não necessitar de pressupostos biológicos, faz-se necessária a sua organização para que seja utilizado em larga escala, além de sua extensão,
resultando em um ambiente em que haja a facilidade de uso, desenvolvimento, manutenção e extensão, além da possibilidade de compartilhamento de
resultados entre os diferentes métodos existentes.

\section{Objetivos} \label{sec:objetivos}

O objetivo deste trabalho é aplicar Engenharia de Software sobre um conjunto de ferramentas e códigos legados desenvolvidos por mim e diversos outros
pesquisadores do FESC (Física Estatística e Sistemas Complexos), desenvolvidos com pouca metodologia, seguido de um levantamento do era necessário para que
se pudesse organizar os \textit{softwares} utilizados objetivando uma melhor execução e experiência de utilização por pesquisadores que detém de pouco
ou quase nenhum conhecimento de desenvolvimento de software e ambientes Linux, além de permitir sua evolução e extensibilidade.

As práticas de Engenharia de Software, como a definição de requisitos, modelagem e implementação do sistema estão entre os objetivos por serem essenciais
para garantir um bom projeto de \textit{software}, que consequentemente possar ser utilizado, mantido, extendido e evoluído facilmente.

\sigla{FESC}{Física Estatística e Sistemas Complexos - Instituto de Física - UFBA}

\section{Contribuições}

Este trabalho irá fornecer um pacote de \textit{software} com uma modelagem bem elaborada, incluindo requisitos e interações entre o usuário
e o sistema (Diagrama de Casos de Uso), organização de dados e informações (Diagrama de Classes) e dinâmica de execução do sistema (Diagramas de Sequência).
O trabalho de modelagem virá acompanhado de um protótipo de \textit{software} que inclui algumas das funcionalidades dispostas nos requisitos.

Podemos dividir as contribuições:

\begin{itemize}
  \item{\textbf{Para os desenvolvedores:} o protótipo, com uma documentação abrangente, permitindo a fácil
manutenção, inclusão das funcionalidades levantadas nos requisitos e ainda não implementadas e extensão do sistema para novas necessidades}
  \item{\textbf{Para os pesquisadores:} um ambiente de fácil utilização e execução de Análise Filogenética através de Redes Complexas, com
geração de gráficos, armazenamento e organização de arquivos, salvamento e carregamento de projetos, possibilitando o uso por pesquisadores
que não são da área de Ciência da Computação.}
\end{itemize}


\section{Estrutura do trabalho}

Esta monografia está organizada da seguinte maneira:

No capítulo \ref{cap:analisefilo} apresenta-se a fundamentação teórica com seus principais conceitos
e os passos para a execução do método de Análise Filogenética através de Redes Complexas: escolha das sequências proteicas,
similaridade, construção e caracterização de redes, limiar crítico e entremeação e centralidade.

No capítulo \ref{cap:navi}, mostra-se o projeto do sistema propriamente dito, e suas questões
de escopo, organização das informações e sua dinâmica de execução.

No capítulo \ref{cap:resultados} são mostrados detalhes de implementação, ambiente de execução, ferramentas de
desenvolvimento, além de discutidos quais os objetivos foram alcançados , e levantadas
outras questões importantes e dificuldades encontradas.

Por fim, é apresentada a conclusão do trabalho e os trabalhos futuros (capítulo \ref{cap:conclusao}), além de anexos e referências bibliográficas.

% use sua propria estrutura

%Problema etc

%Objetivo etc

%Resultados esperados etc

%Estrutura/Organizao etc
