\chapter{Introdução}

A crescente evolução dos computadores e sua capacidade de processamento vem influenciando diretamente as mais diversas áreas do conhecimento. A Biologia
tem aproveitado o crescimento do poder computacional para realizar análises cada vez mais poderosas sobre espécies, organismos e sua carga genética.
A possibilidade de comparação de grandes sequências de DNA em um curto intervalo de tempo possibilita resultados impossíveis de serem obtidos através
de uma comparação manual.

A Bioinformática é hoje uma das áreas multidisciplinares mais promissoras, e a comparação de sequências proteicas permite um impulso na área de Análise
Filogenética, onde diversos métodos tem surgido. Outras áreas do conhecimento que têm se beneficiado bastante com a crescente evolução computacional são a
Física e a Matemática que, assim como a Biologia, podem atingir resultados rapidamente antes impossíveis, e acabam gerando um leque de opções para pesquisa
em computação aplicada.

Hoje em dia a demanda por profissionais e pesquisadores de Ciência da Computação nestas áreas tem se mostrado muito grande, e a junção de duas ou mais delas
em um contexto multidisciplinar tem guiado as pesquisas a resultados muito interessantes, além de gerar produtos e também ideias para novos projetos de
pesquisa. Em um contexto de Física Estatística e Sistemas Complexos, aliando-se conceitos matemáticos da Teoria dos Grafos, foi possível o
desenvolvimento de um método para Análise Filogenética, que levou a resultados comparáveis com os de outros métodos existentes \cite{andrade2011}.

Nos três anos de pesquisa e desenvolvimento do método, houve uma maior preocupação com resultados, o que fez com que o processo de desenvolvimento de
artefatos computacionais  para que se chegasse a eles não seguisse um metodologia de desenvolvimento baseada em conceitos de Engenharia de Software. Como
resultado desse processo de construção de software sem metodologia, o ambiente computacional gerado necessita de
uma curva de aprendizado grande para a sua execução, além de cuidados com relação à organização e nomenclatura de arquivos, configurações,
formatação de dados de entrada e saída para o uso em diferentes programas. A partir do momento em que o método está concluído, testado e comparado com
outros métodos, se faz necessário sua organização para diminuir esta curva de aprendizado. A aplicação de uma metodologia de Engenharia de Software ao
sistema atual se dá a partir de (i) concepção de uma arquitetura que permita uma separação em camadas de interação com o usuário, lógica de negócio e
persistência; (ii) controle de execução do processo; (iii) possibilidade de extensibilidade; (iv) utilização de código legado.

\section{Motivação}

A proposta deste trabalho surgiu a partir do início de 2007, quando o grupo de pesquisa Física Estatística e Sistemas Complexos (FESC),
do Instituto de Física da Universidade Federal da Bahia, resolveu utilizar sua experiência com Redes Complexas aplicando-a a Análise Filogenética.
Os resultados obtidos com a aplicação do método a dados biológicos abrem espaço para a utilização do mesmo como uma alternativa aos métodos
já existentes no contexto da pesquisa em Análise Filogenética. A partir do momento em que esta alternativa se torna viável, na medida em que traz resultados
válidos e comparáveis aos dos métodos já conhecidos,
faz-se necessária a sua organização para que seja utilizada em larga escala, além de sua extensão,
resultando em um ambiente em que haja a facilidade de uso, desenvolvimento, manutenção e extensão, além da possibilidade de compartilhamento de
resultados entre os diferentes métodos existentes.

\section{Objetivos} \label{sec:objetivos}

O objetivo deste trabalho é aplicar metodologias baseadas em Engenharia de Software de forma a desenvolver um ambiente integrado que agrupe
um conjunto de ferramentas e códigos legados desenvolvidos pelo grupo de pesquisa FESC (Física Estatística e Sistemas Complexos),
utilizando-se de pouca metodologia. Em seguida, um levantamento do que era necessário para que
se pudesse organizar os programas e \textit{scripts} utilizados objetivando uma melhor execução e experiência de utilização (a serem comprovadas após validação)
 por pesquisadores que detêm de
pouco ou quase nenhum conhecimento de desenvolvimento de software e ambientes Linux, além de permitir sua evolução e extensibilidade.

A utilização de uma metodologia tradicional de Engenharia de Software, com a definição de requisitos, modelagem e implementação do sistema são também
objetivos do trabalho, pois são
essenciais para garantir um bom projeto de \textit{software}, que consequentemente possar ser utilizado, mantido, extendido
e evoluído facilmente.

\sigla{FESC}{Grupo de pesquisa em Física Estatística e Sistemas Complexos - Instituto de Física - UFBA}

\section{Contribuições}

Este trabalho propõe o desenvolvimento de um pacote de \textit{software} com uma preocupação na modelagem bem elaborada, incluindo requisitos e
interações entre o usuário
e o sistema (Diagrama de Casos de Uso), organização de dados e informações (Diagrama de Classes) e dinâmica de execução do sistema (Diagramas de Sequência).
O trabalho de modelagem virá acompanhado de um protótipo que inclui algumas das funcionalidades dispostas nos requisitos.

O sistema beneficiará dois grupos de usuários: (i) desenvolvedores, com um modelo de sistema e uma documentação abrangente, permitindo sua
manutenção, inclusão das funcionalidades levantadas nos requisitos e ainda não implementadas e extensão do modelo para novos requisitos; e (ii) pesquisadores,
com um ambiente para utilização e execução de Análise Filogenética através de Redes Complexas, com geração de gráficos, armazenamento e organização de
arquivos, salvamento e carregamento de projetos, direcionado para pesquisadores que não são da área de Ciência da Computação.

\section{Estrutura do trabalho}

Esta monografia está organizada da seguinte maneira:

No capítulo \ref{cap:analisefilo} apresenta-se a fundamentação teórica com seus principais conceitos
e os passos para a execução do método de Análise Filogenética através de Redes Complexas: escolha das sequências proteicas,
similaridade, construção e caracterização de redes, limiar crítico e entremeação, incluindo requisitos e a proposta.

No capítulo \ref{cap:navi}, mostra-se o projeto do sistema propriamente dito, e suas questões
de escopo, organização das informações, casos de uso, modelagem de dados e sua dinâmica de execução através dos diagramas de sequência.

No capítulo \ref{cap:resultados} são mostrados detalhes de implementação, ambiente de execução, ferramentas de
desenvolvimento, além de discutidos quais objetivos foram alcançados, e levantadas outras questões importantes e dificuldades encontradas.

Por fim, é apresentada a conclusão do trabalho e os trabalhos futuros (no capítulo \ref{cap:conclusao}), além de anexos e referências bibliográficas.

% use sua propria estrutura

%Problema etc

%Objetivo etc

%Resultados esperados etc

%Estrutura/Organizao etc
