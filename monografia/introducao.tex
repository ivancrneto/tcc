\chapter{Introdução}

A crescente evolução dos computadores e sua capacidadde de processamento vem influenciando diretamente as mais diversas áreas do conhecimento. A Biologia
vem aproveitado o crescimento do poder computacional para realizar análises cada vez mais poderosas sobre espécies, organismos e sua carga genética.
A possibilidade de comparação de grandes sequências de DNA em um curto intervalo de tempo possibilita resultados impossíveis de serem obtidos através
de uma comparação manual.

A Bioinformática é hoje uma das áreas multidisciplinares mais promissoras, que tem a Análise Filogenética através de comparação de sequências proteicas ou
nucleotídicas, onde surgem diariamente métodos.

\section{Motivação}

A proposta deste trabalho surgiu a partir do início deste projeto em 2007, quando o grupo de Física Estatística e Sistemas Complexos (FESC),
do Instituto de Física da Universidade Federal da Bahia, resolveu utilizar sua experiência com Redes Complexas aplicando-a a Análise Filogenética.
Os resultados obtidos com a aplicação do método a dados biológicos abrem espaço para a utilização do mesmo como uma alternativa aos métodos
já existentes no contexto da pesquisa em Análise Filogenética. A partir do momento em que surge uma alternativa viável a métodos já conhecidos
e bastante aplicados na Biologia com capacidade de mostrar resultados válidos e comparáveis às alternativas existentes, e valendo-se da vantagem
de não necessitar de pressupostos biológicos, faz-se necessária a sua organização para que seja utilizado em larga escala, além de sua extensão,
resultando em um ambiente em que haja a possibilidade de compartilhamento de resultados entre os diferentes métodos.

\section{Objetivos} \label{sec:objetivos}

Texto de objetivos

\section{Contribuições}

Texto de contribuição

\section{Estrutura do trabalho}

Esta monografia está organizada da seguinte maneira:

\begin{itemize}
  \item{No capítulo \ref{cap:analisefilo} apresenta-se a fundamentação teórica com seus principais conceitos
e os passos para a execução do método de Análise Filogenética através de Redes Complexas: escolha das sequências proteicas (seção \ref{sec:escseq}),
similaridade (seção \ref{sec:similaridade}), construção e caracterização de redes (seção \ref{sec:conscarac}), limiar crítico (seção \ref{sec:limcrit})
e entremeação e centralidade (seção \ref{sec:entremeacao}).}
  \item{No capítulo \ref{cap:navi}, mostra-se o projeto do sistema propriamente dito, e suas questões
de escopo (seção \ref{sec:escopo}), organização das informações (seção \ref{sec:organizacao}) e sua dinâmica de execução (seção \ref{sec:dinamica}).}
  \item{No capítulo \ref{cap:resultados} são mostrados detalhes de implementação, ambiente de execução (seção \ref{sec:ambiente}), ferramentas de
desenvolvimento (seção \ref{sec:ferramentas}), além de discutidos quais os objetivos foram alcançados (seção \ref{sec:discussao}) e levantadas
outras questões importantes e dificuldades encontradas (seção \ref{sec:dificuldades})}
  \item{Por fim, é apresentada a conclusão do trabalho e os trabalhos futuros (capítulo \ref{cap:conclusao}), além de anexos e referências bibliográficas.}
\end{itemize}

% use sua propria estrutura

%Problema etc

%Objetivo etc

%Resultados esperados etc

%Estrutura/Organizao etc
