\chapter{Introdução}

A crescente evolução dos computadores e sua capacidade de processamento vem influenciando diretamente as mais diversas áreas do conhecimento. A Biologia
vem aproveitado o crescimento do poder computacional para realizar análises cada vez mais poderosas sobre espécies, organismos e sua carga genética.
A possibilidade de comparação de grandes sequências de DNA em um curto intervalo de tempo possibilita resultados impossíveis de serem obtidos através
de uma comparação manual.

A Bioinformática é hoje uma das áreas multidisciplinares mais promissoras, e a compração de sequências proteicas permite um impulso na área de Análise
Filogenética, onde diversos métodos tem surgido. Outras áreas do conhecimento que têm se beneficiado bastante com a crescente evolução computacional são a
Física e a Matemática que, assim como a Biologia, podem atingir resultados rapidamente antes impossíveis, e acabam gerando um leque de opções para pesquisa
em computação aplicada.

Hoje em dia a demanda por profissionais e pesquisadores de Ciência da Computação nestas áreas tem se mostrado muito grande, e a junção de duas ou mais destas
áreas em um contexto multidisciplinar tem guiado as pesquisas a resultados espetaculares, além de gerar produtos e também ideias para novos projetos de
pesquisa. Em um contexto de Física Estatística e Sistemas Complexos, aliando-se conceitos matemáticos \textbf{[CITAR]} e Teoria dos Grafos, foi possível o
desenvolvimento de um Método para Análise Filogenética.

Em três anos de pesquisa o foco maior foram os resultados, o que fez com que o processo para que se chegasse a eles ficasse prejudicado, necessitando uma
curva de aprendizado grande para a sua execução, além de cuidados com relação a organização e nomenclatura de arquivos, configurações, formatação de dados
de entrada e saída para o uso em diferentes programas. A partir do momento em que o método está concluído, testado e comparado com outros métodos, se faz
necessário sua organização para diminuir esta curva de aprendizado, tanto no desenvolvimento quando na utilização, o que pode ser visto agora nas três
seções que seguem: motivação, objetivos e contribuições.

\section{Motivação}

A proposta deste trabalho surgiu a partir do início deste projeto em 2007, quando o grupo de Física Estatística e Sistemas Complexos (FESC),
do Instituto de Física da Universidade Federal da Bahia, resolveu utilizar sua experiência com Redes Complexas aplicando-a a Análise Filogenética.
Os resultados obtidos com a aplicação do método a dados biológicos abrem espaço para a utilização do mesmo como uma alternativa aos métodos
já existentes no contexto da pesquisa em Análise Filogenética. A partir do momento em que surge uma alternativa viável a métodos já conhecidos
e bastante aplicados na Biologia com capacidade de mostrar resultados válidos e comparáveis às alternativas existentes, e valendo-se da vantagem
de não necessitar de pressupostos biológicos, faz-se necessária a sua organização para que seja utilizado em larga escala, além de sua extensão,
resultando em um ambiente em que haja a facilidade de uso, desenvolvimento, manutenção e extensão, além da possibilidade de compartilhamento de
resultados entre os diferentes métodos existentes.

Diversos trabalhos buscam fazer Análise Filogenética, mas geralmente utilizam de pressupostos biológicos para tal. O método do FESC tem um diferencial,
que é o de somente utilizar as sequências propriamente ditas, o que torna importante sua organização para deixá-lo pronto para utilização.

\section{Objetivos} \label{sec:objetivos}

O objetivo deste trabalho foi fazer um levantamento das pesquisas realizadas nos últimos três anos com Análise Filogenética através de Redes Complexas
por mim e pelo grupo no qual trabalhei, o FESC (Física Estatística e Sistemas Complexos) para entender o que era necessário para que se pudesse organizar
os \textit{softwares} utilizados para se ter uma facilidade maior de execução e utilização por pesquisadores que detém de pouco ou quase nenhum conhecimento
de desenvolvimento de software e ambientes UNIX, além de prover uma facilidade de manutenção e extensibilidade.

Dentre os objetivos, então, estavam a definição de requisitos, modelagem e implementação do sistema, para que qualquer pessoa possa utilizá-lo facilmente e
também qualquer desenvolvedor possa se tornar mantenedor do mesmo facilmente. Com essas ações, cresce bastante a possibilidade de o método e todo o arcabouço
tecnológico que o envolve seja consequentemente sua aceito na comunidade acadêmica.

\sigla{FESC}{Física Estatística e Sistemas Complexos - Instituto de Física - UFBA}

\section{Contribuições}

Este trabalho irá fornecer um pacote de \textit{software} funcional, com uma modelagem bem elaborada, incluindo requisitos e interações entre o usuário
e o sistema (Diagrama de Casos de Uso), organização de dados e informações (Diagrama de Classes) e dinâmica de execução do sistema (Diagramas de Sequência).
O trabalho de modelagem virá acompanhado de um protótipo de \textit{software} que inclui algumas das funcionalidades dispostas nos requisitos.

Podemos dividir as contribuições em duas:

\begin{itemize}
  \item{\textbf{A primeira, para os desenvolvedores:} um protótipo de \textit{software} funcional e com uma documentação abrangente, permitindo a fácil
manutenção, inclusão das funcionalidades levantadas nos requisitos e ainda não implementadas e extensão do sistema para novas necessidades}
  \item{\textbf{A segunda, para os pesquisadores:} um ambiente de fácil utilização e execução de Análise Filogenética através de Redes Complexas, com
geração de gráficos, armazenamento e organização de arquivos, salvamento e carregamento de projetos, incluindo possibilidade de uso por pesquisadores
que não são da área de Ciência da Computação.}
\end{itemize}


\section{Estrutura do trabalho}

Esta monografia está organizada da seguinte maneira:

\begin{itemize}
  \item{No capítulo \ref{cap:analisefilo} apresenta-se a fundamentação teórica com seus principais conceitos
e os passos para a execução do método de Análise Filogenética através de Redes Complexas: escolha das sequências proteicas (seção \ref{sec:escseq}),
similaridade (seção \ref{sec:similaridade}), construção e caracterização de redes (seção \ref{sec:conscarac}), limiar crítico (seção \ref{sec:limcrit})
e entremeação e centralidade (seção \ref{sec:entremeacao}).}
  \item{No capítulo \ref{cap:navi}, mostra-se o projeto do sistema propriamente dito, e suas questões
de escopo (seção \ref{sec:escopo}), organização das informações (seção \ref{sec:organizacao}) e sua dinâmica de execução (seção \ref{sec:dinamica}).}
  \item{No capítulo \ref{cap:resultados} são mostrados detalhes de implementação, ambiente de execução (seção \ref{sec:ambiente}), ferramentas de
desenvolvimento (seção \ref{sec:ferramentas}), além de discutidos quais os objetivos foram alcançados (seção \ref{sec:discussao}) e levantadas
outras questões importantes e dificuldades encontradas (seção \ref{sec:dificuldades})}
  \item{Por fim, é apresentada a conclusão do trabalho e os trabalhos futuros (capítulo \ref{cap:conclusao}), além de anexos e referências bibliográficas.}
\end{itemize}

% use sua propria estrutura

%Problema etc

%Objetivo etc

%Resultados esperados etc

%Estrutura/Organizao etc
