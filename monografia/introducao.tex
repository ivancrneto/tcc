\chapter{Introdução}

Modelo de monografia usando as normas ABNT (Associao Brasileira de Normas
Tenicas) \sigla{ABNT}{Associao Brasileira de Normas Tenicas}
e adaptao personalizada 
do padro do Departamento de Cincia da Computao (DCC) da Universidade
Federal da Bahia (UFBA).
\sigla{DCC}{Departamento de Cincia da Computao}
\sigla{UFBA}{Universidade Federal da Bahia}
Fontes latex cedidos pela ABNT e disponibilizados por 
Maurcio Vieira. (Valeu Maurix!). Adaptado por Abelmon Bastos por solicitao
da \profa\ Dbora Abdalla para o semestre 2005.1.
Adaptado por Rodrigo Rocha por solicitao da \profa\ Dbora Abdalla no fim
do semestre 2007.1.

A crescente evolução dos computadores e sua capacidadde de processamento vem influenciando diretamente as mais diversas áreas do conhecimento. A Biologia
vem aproveitado o crescimento do poder computacional para realizar análises cada vez mais poderosas sobre espécies, organismos e sua carga genética.
A possibilidade de comparação de grandes sequências de DNA em um curto intervalo de tempo possibilitam resultados impossíveis de serem obtidos através
de uma comparação manual.

A Bioinformática é hoje uma das áreas multidisciplinares mais promissoras, que tem a Análise Filogenética através de comparação de sequências proteicas ou
nucleotídicas, onde surgem diariamente métodos.

\section{Motivação}

A proposta deste trabalho surgiu a partir do início deste projeto em 2007, quando o grupo de Física Estatística e Sistemas Complexos (FESC),
do Instituto de Física da Universidade Federal da Bahia, resolveu utilizar sua experiência com Redes Complexas aplicando-a a Análise Filogenética.
Os resultados obtidos com a aplicação do método a dados biológicos abrem espaço para a utilização do mesmo como uma alternativa aos métodos
já existentes no contexto da pesquisa em Análise Filogenética. A partir do momento em que surge uma alternativa viável a métodos já conhecidos
e bastante aplicados na Biologia com capacidade de mostrar resultados válidos e comparáveis às alternativas existentes, e valendo-se da vantagem
de não necessitar de pressupostos biológicos, faz-se necessária a sua organização para que seja utilizado em larga escala, além de sua extensão,
resultando em um ambiente em que haja a possibilidade de compartilhamento de resultados entre os diferentes métodos.

\section{Objetivos}

Texto de objetivos

\section{Contribuições}

Texto de contribuição

\section{Estrutura do trabalho}

Esta monografia está organizado da seguinte maneira: na sessão 2 apresenta-se a fundamentação teórica com seus principais conceitos;]
na sessão 3 mostra-se a proposta deste projeto; as sessões 4 e 5 descrevem trabalhos relacionados, cronograma e resultados esperados,
representando o plano de trabalho.


% use sua propria estrutura

%Problema etc

%Objetivo etc

%Resultados esperados etc

%Estrutura/Organizao etc
