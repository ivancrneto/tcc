\chapter{Conceitos} \label{cap:conceitos}

Este trabalho relata o onon ono non ono non ono non ono non ono non on.

\section{Listagens} \label{sec:listagens}

Na listagem \ref{lst:testjUnit} 
e mostrado o teste do metodo \texttt{Engine.initialize()}:

\lstset{language=java}
\lstset{commentstyle=\textit}
\begin{lstlisting}[frame=trbl, caption=Classe Factory2D,label=lst:testjUnit]{}
public class EngineTest
// JUnitDoclet begin extends_implements
extends TestCase
// JUnitDoclet end extends_implements
{
  //...
  public void testInitialize() throws Exception {
   // JUnitDoclet begin method initialize
   EngineState engineState = (EngineState) PrivateAccessor.
    getField(engine,"engineState");
   engine.initialize();
   assertEquals(engineState, new InitState());
   // JUnitDoclet end method initialize
  }
  ...
}
\end{lstlisting}

\section{Figuras} \label{sec:figuras}

E possivel usar imagens vetoriais no formato PDF, como pode ser visto
na figura \ref{fig:ufba}, ou imagens \emph{bitmap} no formato PNG, como
a da figura \ref{fig:ufba2}.

\begin{figure}
\centering
\includegraphics{brasaoUFBA2}
\caption{Brasao da UFBA (vetorial)}
\label{fig:ufba}
\end{figure}

\begin{figure}
\centering
\includegraphics[width=0.3\textwidth]{brasaoUFBA}
\caption{Brasao da UFBA (\emph{bitmap})}
\label{fig:ufba2}
\end{figure}


\chapter{Outro capitulo}
\label{cap:outrocapitulo}

Como visto no capitulo \ref{cap:conceitos}, no nonno\footnote{Isto e uma nota
de rodape.} no nonon onono:
\begin{itemize}
  \item{nononoo}
  \item{nononono}
  \item{no}
\end{itemize}


Nonono nonnono onononono \cite{fowler2000}.
