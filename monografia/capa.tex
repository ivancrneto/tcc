% Capa com Braso

\begin{titlepage}
   \begin{center}
    	%logotipo
               \includegraphics{brasaoUFBA} \\
	%\vspace{0.7in}
              \centering{ 
	      \bf{
	      \LARGE{
		\uppercase{UNIVERSIDADE FEDERAL DA BAHIA} \\
 	      }
	      \Large {
                   	\uppercase{INSTITUTO DE MATEMÁTICA} \\
	      }
                   \large {
                       \uppercase{DEPARTAMENTO DE CIÊNCIA DA COMPUTAÇÃO} \\
                  }
              } }
   \end{center}
   \vfill
   \begin{center}
       \bf{
       \large{\uppercase{\meunome}  \\  }
       }
   \end{center}
   \vspace{0.2in}
   \begin{center}
       \bf{
      	 \LARGE{ \uppercase{\meutitulo} } \\
      	 %\Large{ \uppercase{\meusubtitulo} }
         \obs{\\ \Large{Esta verso da monografia contm comentrios do autor.
          Para remov-los, redefina o comando LaTeX \texttt{obs}.}}
       }
   \end{center}

   \vfill
   \hspace{\stretch{1}}
   \vfill
   \begin{center}
      \normalsize{
          Salvador \\
          \meuano
       }
   \end{center}

\end{titlepage}

%comando abaixo cria uma capa redundante, mas como a capa com braso foi 
% feita 'manualmente', no faz sentido usar este comando:
%\capa

