% vim: tw=80 noai
\documentclass[normaltoc,capchap,capsec,times]{abnt}
%\usepackage{t1enc}
%\usepackage[latin1]{inputenc}
%\usepackage[T1]{fontenc}
\usepackage[utf8]{inputenc}
\usepackage[brazil]{babel}
\usepackage[alf]{abntcite}
\usepackage[ordem=alf]{tabela-simbolos}
\usepackage{url}
\usepackage{graphicx}
\usepackage{listings}
\usepackage{verbatim}
\usepackage{subfigure}
\usepackage{multicol}
\usepackage{framed}
\usepackage{glossaries}
\def\lstlistingname{Listagem}

\makeglossaries

%%%%%%%%%%%%%%%%%%%%%%%%%%%%%%%%%%%%%%%%%%%%%%%%%%%%
% Dados da monografia
%%%%%%%%%%%%%%%%%%%%%%%%%%%%%%%%%%%%%%%%%%%%%%%%%%%%

\newcommand{\meunome}{Ivan Carmo da Rocha Neto}
\newcommand{\meutitulo}{AIAF: Construção de um Ambiente Integrado de apoio à Análise Filogenética através de Redes Complexas}
%\newcommand{\meusubtitulo}{Uma abordagem focada na experiência do usuário}
\newcommand{\meuano}{2011.2}
\newcommand{\meuorientador}{Orientador: \prof\ Antonio Lopes Apolinário Junior.}
\newcommand{\meucoorientador}{Co-orientador: \prof\ Roberto Fernandes Silva Andrade.}

%%%%%%%%%%%%%%%%%%%%%%%%%%%%%%%%%%%%%%%%%%%%%%%%%%%%

%% O comando \obs aqui definido permite que o autor faca anotacoes na
%% monografia que aparecem no PDF gerado. Para ativar o comando, descomente
%% a primeira linha e comente a segunda.
%% Exemplo de uso: \obs{Preciso melhorar este pargrafo...}

%\newcommand{\obs}[1]{\underline{\textbf{OBSERVAO}}: \emph{#1}}
\newcommand{\obs}[1]{}

\def\ordfem{\mbox{\raise .35em \hbox{\underline{\scriptsize a}\ }}}
\def\ordmasc{\mbox{\raise .35em \hbox{\underline{\scriptsize o}\ }}}
\def\profa{Prof\ordfem.}
\def\prof{Prof\ordmasc.}

%%%%%%%%%%%%%%%%%%%%%%%%%%%%%%%%%%%%%%%%%%%%%%%%%%%%

\begin{document}

% Capa com Braso

\begin{titlepage}
   \begin{center}
    	%logotipo
               \includegraphics{brasaoUFBA} \\
	%\vspace{0.7in}
              \centering{ 
	      \bf{
	      \LARGE{
		\uppercase{UNIVERSIDADE FEDERAL DA BAHIA} \\
 	      }
	      \Large {
                   	\uppercase{INSTITUTO DE MATEMÁTICA} \\
	      }
                   \large {
                       \uppercase{DEPARTAMENTO DE CIÊNCIA DA COMPUTAÇÃO} \\
                  }
              } }
   \end{center}
   \vfill
   \begin{center}
       \bf{
       \large{\uppercase{\meunome}  \\  }
       }
   \end{center}
   \vspace{0.2in}
   \begin{center}
       \bf{
      	 \LARGE{ \uppercase{\meutitulo} } \\
      	 %\Large{ \uppercase{\meusubtitulo} }
         \obs{\\ \Large{Esta verso da monografia contm comentrios do autor.
          Para remov-los, redefina o comando LaTeX \texttt{obs}.}}
       }
   \end{center}

   \vfill
   \hspace{\stretch{1}}
   \vfill
   \begin{center}
      \normalsize{
          Salvador \\
          \meuano
       }
   \end{center}

\end{titlepage}

%comando abaixo cria uma capa redundante, mas como a capa com braso foi 
% feita 'manualmente', no faz sentido usar este comando:
%\capa



%\folhaderosto
% o comando acima foi comentado para no criar uma folha de 
% rosto redundante, j que ela feita 'manualmente' abaixo

\begin{titlepage}
 \vfill
 \begin{center}
   {\large \uppercase{ \bf{ \meunome\ } } } \\[7cm]
   {\Huge \uppercase{ \bf{ \meutitulo\ } } }\\[1cm]
   \vfill
   \hspace{.45\textwidth} % posicionando a minipage
   \begin{minipage}{.5\textwidth}
     \begin{espacosimples}
       \bf{
	Monografia apresentada ao Curso de gradução em Ciência da Computação, 
	Departamento de Ciência da Computação, Instituto de Matemática, Universidade Federal da 
	Bahia, como requisito parcial para obtenção do grau de Bacharel em Ciência da Computação. \\ 
       }      
     \end{espacosimples}
     \begin{espacosimples}    
       \meuorientador
       \newline
       \meucoorientador
     \end{espacosimples}
   \end{minipage}
   \vfill
   Salvador \\
   \meuano
 \end{center}
\end{titlepage}


\begin{resumo}

O método de análise filogenética através de redes complexas parte de princípios teóricos da Física Estatística e utiliza técnicas computacionais
baseadas em Teoria dos Grafos. Essa abordagem se mostra como uma alternativa viável aos pesquisadores, na medida em que traz resultados comparáveis aos
de outros métodos. Neste trabalho de análise e desenvolvimento são organizados os dados, informações e programas desenvolvidos pelo grupo
de pesquisa FESC (Física Estatística e Sistemas Complexos - Instituto de Física - UFBA), e é projetado um \textit{software} visando a facilidade de uso
por biólogos, físicos e bioinformáticos como uma importante alternativa às ferramentas existentes para análise filogenética. Os artefatos originais foram
desenvolvidos com falta de integração e uma forma de organização que dificulta a utilização por parte dos pesquisadores. O produto deste trabalho é um
ambiente integrado de controle e gerenciamento de artefatos, com a utilização de persistência e modelado de forma que permita extensibilidade.


\textbf{Palavras-chave:}
análise filogenética,
engenharia de software, 
reengenharia,
modelagem,
desenvolvimento.
\end{resumo}

% O abstract e' opcional.
\begin{abstract}
The method of phylogenetic analysis through complex networks comes from the theoretical principles of Statistical Physics and uses computational
techniques based on Graph Theory. There is an evidence that this approach is a viable alternative to the researchers,
because it has results that are comparable to other methods. In this work of analysis and development data, information and programs developed by FESC
(Group of Statistical Physics
and Complex Systems - Physics Institute - UFBA) are organized, and it is designed a software that is easily useful
 by biologists, physicists and bioinformaticians
as an important alternative of existing tools for phylogenetic analysis. The original artifacts were developed with a lack of integration and form of
organization that makes it difficult to use by researchers. The product of this work is an integrated control and
management of artifacts, with the use of persistence and modeled to allow extensibility.

\textbf{Keywords:} 
phylogenetic analysis,
software engineering,
reengineering,
modeling,
development
\end{abstract}

%% As listas a seguir sao opcionais:
\listadefiguras
%\listadetabelas
\listadesiglas
%\listadesimbolos

\sumario

% O conteudo da monografia esta' nos seguintes arquivos:
\chapter{Introdução}

A crescente evolução dos computadores e sua capacidade de processamento vem influenciando diretamente as mais diversas áreas do conhecimento. A Biologia
tem aproveitado o crescimento do poder computacional para realizar análises cada vez mais poderosas sobre espécies, organismos e sua carga genética.
A possibilidade de comparação de grandes sequências de DNA em um curto intervalo de tempo possibilita resultados impossíveis de serem obtidos através
de uma comparação manual.

\newglossaryentry{engsoft}{name=Engenharia de Software,
description={\hspace{2 mm}área do conhecimento da computação voltada para a especificação, desenvolvimento e manutenção de sistemas de software
aplicando tecnologias e práticas de gerência de projetos e outras disciplinas, objetivando organização, produtividade e qualidade}}

\newglossaryentry{bioinformatica}{name=Bioinformática,
description={\hspace{2 mm}corresponde a aplicação das técnicas da Informática, no sentido de análise da informação na área de estudo da Biologia, utilizando
tais técnicas computacionais e matemáticas à geração e gerenciamento de bioinformação}}

A \gls{bioinformatica} é hoje uma das áreas multidisciplinares mais promissoras, e a comparação de sequências proteicas permite um impulso na área de Análise
Filogenética, onde diversos métodos tem surgido. Outras áreas do conhecimento que têm se beneficiado bastante com a crescente evolução computacional são a
Física e a Matemática que, assim como a Biologia, podem atingir resultados rapidamente antes impossíveis, e acabam gerando um leque de opções para pesquisa
em computação aplicada.

\newglossaryentry{fisica-estatistica}{name=Física Estatística,
description={\hspace{2 mm}é o ramo da Física que usa métodos da teoria das probabilidades e estatística e, particularmente, as ferramentas matemáticas para
lidar com grandes populações e aproximações, na solução de problemas físicos. Seu principal objetivo é esclarecer as propriedades da matéria sob conjuntos,
em termos de leis físicas que regem o movimento atômico}}

\newglossaryentry{sistemas-complexos}{name=Sistemas Complexos,
description={\hspace{2 mm}são sistemas onde suas propriedades não são uma consequência natural de seus elementos constituintes vistos isoladamente. As
propriedades emergentes de um sistema complexo decorrem em grande parte da relação não-linear entre as partes}}

Hoje em dia a demanda por profissionais e pesquisadores de Ciência da Computação nestas áreas tem se mostrado muito grande, e a junção de duas ou mais delas
em um contexto multidisciplinar tem guiado as pesquisas a resultados muito interessantes, além de gerar produtos e também ideias para novos projetos de
pesquisa. Em um contexto de \gls{fisica-estatistica} e \gls{sistemas-complexos}, aliando-se conceitos matemáticos da Teoria dos Grafos, foi possível o
desenvolvimento de um método para Análise Filogenética, que levou a resultados comparáveis com os de outros métodos existentes \cite{andrade2011}.


\newglossaryentry{artefatos}{name=artefatos computacionais,
description={\hspace{2 mm}são vários tipos de subprodutos concretos produzido durante o desenvolvimento de software. Alguns artefatos (por exemplo,
casos de uso, diagramas de classes e outros modelos UML, requisitos e documentos de projeto) ajudam a descrever a função, arquitetura e o design do software.
Outros artefatos estão relacionados com o próprio processo de desenvolvimento - tais como planos de projetos, processos de negócios e avaliações de risco.
Podem ser manuais, arquivos executáveis, módulos etc}}


\newglossaryentry{persistencia}{name=persistência,
description={\hspace{2 mm}refere-se ao armazenamento não-volátil de dados, por exemplo, o armazenamento em um dispositivo físico como um disco rígido.
Pode-se dizer que de maneira geral, o termo persistência é associado a uma ação que consiste em manter em meio físico recuperável, como banco de dados
ou arquivo, de modo a garantir a permanência das informações de um determinado estado de um objeto lógico.}}

Nos três anos de pesquisa e desenvolvimento do método, houve uma maior preocupação com resultados, o que fez com que o processo de desenvolvimento de
\gls{artefatos} para que se chegasse a eles não seguisse um metodologia de desenvolvimento baseada em conceitos de \gls{engsoft}. Como
resultado desse processo de construção de software sem metodologia, o ambiente computacional gerado necessita de
uma curva de aprendizado grande para a sua execução, além de cuidados com relação à organização e nomenclatura de arquivos, configurações,
formatação de dados de entrada e saída para o uso em diferentes programas. A partir do momento em que o método está concluído, testado e comparado com
outros métodos, se faz necessário sua organização para diminuir esta curva de aprendizado. A aplicação de uma metodologia de Engenharia de Software ao
sistema atual se dá a partir de (i) concepção de uma arquitetura que permita uma separação em camadas de interação com o usuário, gerenciamento de
e controle, e \gls{persistencia}; (ii) controle de execução do processo; (iii) possibilidade de extensibilidade; (iv) utilização de código legado.

\section{Motivação}

A proposta deste trabalho surgiu a partir do início de 2007, quando o grupo de pesquisa Física Estatística e Sistemas Complexos (FESC),
do Instituto de Física da Universidade Federal da Bahia, resolveu utilizar sua experiência com Redes Complexas aplicando-a a Análise Filogenética.
Os resultados obtidos com a aplicação do método a dados biológicos abrem espaço para a utilização do mesmo como uma alternativa aos métodos
já existentes no contexto da pesquisa em Análise Filogenética. A partir do momento em que esta alternativa se torna viável, na medida em que traz resultados
válidos e comparáveis aos dos métodos já conhecidos,
faz-se necessária a sua organização para que seja utilizada em larga escala, além de sua extensão,
resultando em um ambiente em que haja a facilidade de uso, desenvolvimento, manutenção e extensão, além da possibilidade de compartilhamento de
resultados entre os diferentes métodos existentes.

\section{Objetivos} \label{sec:objetivos}

O objetivo deste trabalho é aplicar metodologias baseadas em Engenharia de Software de forma a desenvolver um ambiente integrado que agrupe
um conjunto de ferramentas e códigos legados desenvolvidos pelo grupo de pesquisa FESC (Física Estatística e Sistemas Complexos),
utilizando-se de pouca metodologia. Em seguida, um levantamento do que era necessário para que
se pudesse organizar os programas e \textit{scripts} utilizados objetivando melhorar sua execução e experiência de utilização (a serem comprovadas após validação)
 por pesquisadores que detêm de
pouco ou quase nenhum conhecimento de desenvolvimento de software e ambientes Linux, além de permitir sua evolução e extensibilidade.


\newglossaryentry{modelagem}{name=modelagem,
description={\hspace{2 mm}atividade de construir modelos que expliquem as características ou o comportamento de um \textit{software}.
Na construção do \textit{software} os modelos podem ser usados na identificação das características e funcionalidades que o \textit{software} deverá prover
(análise de requisitos), e no planejamento de sua construção}}


A utilização de uma metodologia com a definição de requisitos, \gls{modelagem} e criação de artefatos e implementação do sistema são também
objetivos do trabalho, pois são
essenciais para garantir um bom projeto de \textit{software}, que consequentemente possar ser utilizado, mantido, extendido
e evoluído facilmente.

\sigla{FESC}{Grupo de pesquisa em Física Estatística e Sistemas Complexos - Instituto de Física - UFBA}

\section{Contribuições}

\newglossaryentry{diagrama-casos-uso}{name=Diagrama de Casos de Uso,
description={\hspace{2 mm}descreve a funcionalidade proposta para um novo sistema que será projetado. Pode ser tido como um
documento narrativo que descreve a sequência de eventos de um ator que usa um sistema para completar um processo}}

\newglossaryentry{diagrama-classes}{name=Diagrama de Classes,
description={\hspace{2 mm}é uma representação da estrutura e relações das classes que servem de modelo para objetos. É uma modelagem muito útil para o
sistema, define todas as classes que o sistema necessita possuir e é a base para a construção dos diagramas de comunicação, sequência e estados}}

\newglossaryentry{diagrama-sequencia}{name=Diagrama de Sequências,
description={\hspace{2 mm}descreve a maneira como os grupos de objetos colaboram em algum comportamento ao longo do tempo. Ele registra o comportamento
de um único caso de uso e exibe os objetos e as mensagens passadas entre esses objetos no caso de uso}}

Este trabalho propõe o desenvolvimento de um pacote de \textit{software} com uma preocupação na modelagem bem elaborada, incluindo requisitos e
interações entre o usuário
e o sistema (\gls{diagrama-casos-uso}), organização de dados e informações (\gls{diagrama-classes}) e dinâmica de execução do sistema (\gls{diagrama-sequencia}).
O trabalho de modelagem virá acompanhado de um protótipo que inclui algumas das funcionalidades dispostas nos requisitos.

O sistema beneficiará dois grupos de usuários: (i) desenvolvedores, com um modelo de sistema e uma documentação abrangente, permitindo sua
manutenção, inclusão das funcionalidades levantadas nos requisitos e ainda não implementadas e extensão do modelo para novos requisitos; e (ii) pesquisadores,
com um ambiente integrado que permitirá a utilização e execução de Análise Filogenética através de Redes Complexas por uma interface gráfica,
armazenamento e organização de
arquivos, salvamento e retomada de execuções das análises de forma transparente, direcionado para pesquisadores que não são da área de Ciência da Computação.

\section{Estrutura do trabalho}

Esta monografia está organizada da seguinte maneira:

No capítulo \ref{cap:analisefilo} apresenta-se a fundamentação teórica com seus principais conceitos
e os passos para a execução do método de Análise Filogenética através de Redes Complexas: escolha das sequências proteicas,
similaridade, construção e caracterização de redes, limiar crítico e entremeação, incluindo requisitos e a proposta.

No capítulo \ref{cap:navi}, mostra-se o projeto do sistema propriamente dito, e suas questões
de escopo, organização das informações, casos de uso, modelagem de dados e sua dinâmica de execução através dos diagramas de sequência.

No capítulo \ref{cap:resultados} são mostrados detalhes de implementação, ambiente de execução, ferramentas de
desenvolvimento, além de discutidos quais objetivos foram alcançados, e levantadas outras questões importantes e dificuldades encontradas.

Por fim, é apresentada a conclusão do trabalho e os trabalhos futuros (no capítulo \ref{cap:conclusao}), além de anexos e referências bibliográficas.

% use sua propria estrutura

%Problema etc

%Objetivo etc

%Resultados esperados etc

%Estrutura/Organizao etc

\chapter{Análise Filogenética através de Redes Complexas} \label{cap:analisefilo}

Análise Filogenética é uma área da Biologia que tem despertado bastante interesse pelos pesquisadores. A evolução do poder de processamento dos computadores
permitiu que se pudesse comparar sequências proteicas ou nucleotídicas muito rapidamente, gerando um poder de análise abrangente no que tange as relações entre
proteinas, enzimas, rotas metabólicas e consequentemente seres vivos. Essas análises, por conseguinte, geraram descobertas importantes que influenciaram o
modo como enxergamos a evolução das espécies e a relação entre elas em geral.

Atualmente existem quatro métodos bem aceitos na literatura para Análise Filogenética: Análise Bayesiana \textbf{[CITAR]}, Análise de Parcimônia
\textbf{[CITAR]}, Análise de Distâncias \textbf{[CITAR]} e Análise por Verossimilhança \textbf{[CITAR]}. Estas análises partem de alguns pressopostos
biológicos, que são informações iniciais necessárias a sua realização. O uso da Física Estatística para tratar os dados referentes a proteinas de
determinadas rotas metabólicas permite o desenvolvimento de um método que não necessite desses pressupostos.

A área de Física Estatística atualmente apresenta forte caráter interdisciplinar, permitindo a modelagem e análise de diversos sistemas
que englobam outras áreas de conhecimento. A aplicação da Teoria das Redes Complexas em Bioinformática, apoiada pela Teoria dos Grafos,
juntamente com conhecimentos de Física, Matemática e Biologia vem se mostrando como uma alternativa interessante na modelagem de sistemas biológicos,
e serviu para que o FESC (Física Estatística e Sistemas Complexos - Instituto de Física - UFBA) desenvolvesse um método para Análise Filogenética.

\sigla{FESC}{Física Estatística e Sistemas Complexos - Instituto de Física - UFBA}

Como foi dito anteriormente, o objetivo deste trabalho é controlar o processo e fornecer um ambiente para o armazenamento de informações em relação às
execuções, além de uma interface de fácil utilização para um pesquisador da área de Biologia. A figura \ref{fig:fluxograma} mostra o fluxograma geral
sobre os passos da execução do método. As seções que seguem explicam estes passos com mais detalhes.

\begin{figure}
\centering
\includegraphics{brasaoUFBA2}
\caption{Fluxograma com os passos da execução do método de análise filogenética desenvolvido pelo FESC.}
\label{fig:fluxograma}
\end{figure}


\section{Escolha de Sequências Proteicas} \label{sec:escseq}

A primeira etapa do método consiste na escolha das sequências que serão utilizadas para a construção e caracterização da rede.
Usualmente a escolha é feita a partir de um banco de dados biológico, disponível livremente pela web. Durante todo o trabalho na utilização
do método, as sequências escolhidas vieram do NCBI \cite{ncbi}, um banco de dados bastante utilizado no meio biológico, que contém dados de sequências
nucleotídicas e sequências proteicas, entre várias outras informações. A escolha das sequências proteicas é feita utilizando como critério
uma rota metabólica, separando as sequências pela enzima que as produzem na rota.

\sigla{NCBI}{National Center for Biotechnology Information}

Esta é uma etapa manual, onde são feitas buscas por data e são realizadas determinadas filtragens no site do NCBI para obter as sequências desejadas. O
sistema desenvolvido neste trabalho leva em consideração que as sequências já foram escolhidas e as recebe como entrada inicial. O resultado desta etapa
são arquivos no formato GenBank (NCBI) que são filtrados e organizados para facilitar a identificação de organismos em etapas posteriores e proporcionar
a execução da similaridade.

\section{Similaridade} \label{sec:similaridade}

Para executar a similaridade, primeiro os dados no formato GenBank são filtrados e armazenados em um banco de dados relacional. Então é utilizado o BLAST
(Basic Alignment Search Tool) \cite{blast1997} para comparação entre as sequências. O BLAST trabalha tanto com sequências nucleotídicas quanto proteicas.
O interesse do uso do BLAST para o método em questão está na comparação de sequências proteicas utilizando o alinhamento local, que compara duas sequências
isoladamente e retorna uma porcentagem de similaridade entre elas.

\sigla{BLAST}{Basic Alignment Search Tool}

A partir do resultado do BLAST, pode-se construir uma matriz de similaridades \cite{andrade2006}, onde as linhas e colunas são as sequências e
a relação é a similaridade entre elas. Desta matriz são construídas as redes que caracterizam as sequências, que serão descritas na próxima seção.

\section{Construção e Caracterização das Redes} \label{sec:conscarac}

O conceito de rede está associado ao de um grafo. É um conjunto de vértices e arestas conectando-os. Podemos dizer que o estudo de uma rede é o
estudo de um grafo, mas em uma escala maior. Redes podem ser tidas como grafos muito grandes e complexos, em que se torne mais preciso o uso de
técnicas estatísticas para a sua análise \cite{bessa2008}. Na literatura de Redes Complexas, em geral, precisamos encontrar uma rede de determinado
conjunto que melhor caracterize este conjunto. Essa escolha se dá pela relação entre ruído e informação \cite{barabasi2004}.

Partindo da matriz de similaridades, podemos construir até 101 matrizes de adjacência, se levarmos em consideração que podemos escolher um valor
limiar e construir a rede baseada nele. Por exemplo, se escolhermos o valor 60 como limiar, construiremos uma rede onde haverá uma aresta aresta entre
dois vértices (sequências) se e somente se a similaridade entre eles em questão for maior ou igual a 60. Uma rede com limiar muito baixo apresenta muito
ruído e uma rede com limiar muito alto apresenta pouca informação.

Partindo das matrizes de adjacência podemos também criar matrizes de vizinhança \cite{andrade2009}, que podem ser obtidas calculando-se os caminhos mínimos
\cite{bessa2008} entre os vértices da rede ou através da multiplicação booleana de matrizes de adjacência de ordens consecutivas \cite{andrade2006}. Na figura 
\ref{fig:matriz-vizinhanca}, temos um exemplo de matriz de vizinhança (com os números em forma de cor).

\begin{figure}
\centering
\includegraphics[scale=0.53]{matriz-vizinhanca}
\caption{Exemplo de Matriz de Vizinhança no formato de Matriz de Cores.}
\label{fig:matriz-vizinhanca}
\end{figure}

Para a execução desta etapa, é utilizado o programa \textit{Rede Crítica}, desenvolvido na linguagem Fortran. As matrizes de vizinhança são importantes para
etapas posteriores do processo, inclusive a etapa seguinte, que é a determinação do limiar crítico.

\section{Limiar Crítico} \label{sec:limcrit}

Já se sabe que o limiar crítico de uma rede revela a melhor relação entre ruído e informação. Existem diversas maneiras de se determinar o limiar
crítico da rede. Uma delas, utilizada pelo FESC, é o método das distâncias \textbf{[CITAR REFERÊNCIA DO ARTIGO DA PLOS]}. Nele, calcula-se a
distância euclidiana entre matrizes de vizinhança de limiares consecutivos. A matriz que apresentar a maior distância entre a de limiar conscutivo
ao seu será a escolhida como a matriz do limiar crítico, para representar a rede. É necessário escolher a melhor rede que representa o conjunto.
A rede do limiar crítico cumpre bem este papel.

Para a execução desta etapa, é utilizado o programa \textit{Rede Crítica}, desenvolvido na linguagem Fortran. Na figura \ref{fig:distancia}, temos o
exemplo de um resultado do cálculo das distâncias entre matrizes de vizinhança, onde o limiar crítico é representado pelo pico do gráfico, ou seja,
o valor 51. A rede que representa todo o conjunto é então a rede do limiar 51, que será utilizada na última etapa do processo, que é a clusterização,
que utiliza um conceito chamado entremeação.

\begin{figure}
\centering
\includegraphics[scale=0.58]{distancia}
\caption{Resultado da execução do método das distâncias entre limiares consecutivos, resultando em um gráfico Distância x Limiar.}
\label{fig:distancia}
\end{figure}

\section{Entremeação e Centralidade} \label{sec:entremeacao}

A Matriz de Vizinhança da rede do limiar crítico é a entrada necessária para a realização da clusterização, que é feita através do método de detecção
de comunidades de Newman e Girvan \cite{newman2004} para identificar a estrutura modular da rede. O método consiste em determinar a aresta mais importante
da rede – aquela por onde passa a maior quantidade de caminhos mínimos de cada par de vértices por toda a rede – e eliminá-la, repetindo a operação
até que não existam mais arestas na rede.

Para a execução desta etapa, é utilizado o programa \textit{Dendo}, em Fortran. Conforme a operação é feita, é possível construir um histograma de remoção
de arestas (dendrograma), que auxilia na identificação de comunidades e propicia as análises biológicas dos resultados, além da comparação com outros
métodos da Biologia. A figura \ref{fig:dendrograma} mostra com detalhes um exemplo de dendrograma para o limiar crítico 51.\newline

\begin{figure}
\centering
\includegraphics[scale=0.73]{dendrograma}
\caption{Exemplo de dendrograma para o limiar crítico 51.}
\label{fig:dendrograma}
\end{figure}

O método em questão utiliza uma determinada quantidade de programas, cada um seguindo certos padrões de entrada e saída, além de arquivos de configuração.
Tais características dificultam sua utilização por pesquisadores de outras áreas que não Física ou Ciência da Computação, além de trazê-los uma preocupação
desnecessária com localização e distribuição de arquivos. Para um melhor aproveitamento do método, seria mais interessante que o controle do processo de
execução fosse realizado por um sistema, dispondo de um ambiente de fácil utilização e que realizasse o gerenciamento das informações por meio de uma camada
de persistência, deixando a questão da organização dos dados transparente para o pesquisador.

A tabela \ref{tab:requisitos} mostra os requisitos necessários para a construção do Navi - Sistema para Gerenciamento de Informações e Execuções de Análise
Filogenética Através de Redes Complexas.

\begin{table}
\centering
\caption{Tabela de requisitos do Navi} % igual ao ambiente figura
\begin{tabular}{c} % com este comando dizemos quantas colunas terá nossa tabela e a posição do texto dentro de cada coluna. Aqui temos três colunas (pois são três "c" dentre {}) e o texto estará centralizado em todas elas (indicado pelo "c", se quisermos alinhados à esquerda "l" ou direita "r"
\hline 
%Requisito & Pontos & Classificação \\
Requisito \\ 
\hline
\hline
%Cruzeiro & 52 & 1 \\
%Sao Paulo & 50 & 2 \\
%Gremio Barueri & 47 & 3 \\
Criar novo projeto \\
Abrir projeto existente \\
Gerar matriz de adjacências de um limiar qualquer \\
Gerar matriz de vizinhança de um limiar qualquer \\
Visualizar o gráfico de matriz de vizinhança qualquer em forma de matriz de cores \\
Analisar limiares \\
Executar clusterização \\
Comparar/executar congruência de redes de diferentes projetos \\
Visualizar gráfico completo da rede \\
Executar congruência em relação a um arquivo cadastrado em um formato padrão \\
Exportar dados do projeto para gráficos em um formato de programas plotadores \\
\hline
\end{tabular}
\label{tab:requisitos}
\end{table} 

\chapter{Navi - Sistema para Gerenciamento de Informações e Execuções de Análise Filogenética Através de Redes Complexas}
\label{cap:navi}

\section{Escopo do Sistema} \label{sec:escopo}

bla

\section{Organização das Informações} \label{sec:organizacao}

bla

\section{Dinâmica do Sistema} \label{sec:dinamica}

bla


\chapter{Resultados}
\label{cap:resultados}


\section{Ambiente de execução} \label{sec:ambiente}

bla
troca de banco de dados relacional para serialização de objetos

\section{Ferramentas de Desenvolvimento} \label{sec:ferramentas}

bla

\section{Discussão} \label{sec:discussao}

bla

\section{Dificuldades} \label{sec:dificuldades}

bla



\section{Listagens} \label{sec:listagens}

Nonono nonnono onononono \cite{fowler2000}.

Na listagem \ref{lst:testjUnit} 
e mostrado o teste do metodo \texttt{Engine.initialize()}:

\lstset{language=java}
\lstset{commentstyle=\textit}
\begin{lstlisting}[frame=trbl, caption=Classe Factory2D,label=lst:testjUnit]{}
public class EngineTest
// JUnitDoclet begin extends_implements
extends TestCase
// JUnitDoclet end extends_implements
{
  //...
  public void testInitialize() throws Exception {
   // JUnitDoclet begin method initialize
   EngineState engineState = (EngineState) PrivateAccessor.
    getField(engine,"engineState");
   engine.initialize();
   assertEquals(engineState, new InitState());
   // JUnitDoclet end method initialize
  }
  ...
}
\end{lstlisting}

Como visto no capitulo \ref{cap:analisefilo}, no nonno\footnote{Isto e uma nota
de rodape.} no nonon onono:
\begin{itemize}
  \item{nononoo}
  \item{nononono}
  \item{no}
\end{itemize}


\section{Figuras} \label{sec:figuras}

E possivel usar imagens vetoriais no \cite{andrade2006} formato PDF, como pode ser visto
na figura \ref{fig:ufba}, ou imagens \emph{bitmap} no formato PNG, como
a da figura \ref{fig:ufba2}.

\begin{figure}
\centering
\includegraphics{brasaoUFBA2}
\caption{Brasao da UFBA (vetorial)}
\label{fig:ufba}
\end{figure}

\begin{figure}
\centering
\includegraphics[width=0.3\textwidth]{brasaoUFBA}
\caption{Brasao da UFBA (\emph{bitmap})}
\label{fig:ufba2}
\end{figure}

\chapter{Conclusão} \label{cap:conclusao}

Com a evolução do poder dos computadores, criação de novas tecnologias, especialização dos estudantes, pesquisadores e profissionais, a Ciência da Computação
passa a exercer um papel cada vez mais relevante na sociedade, seja na indústria, no mercado ou na pesquisa. Com o passar dos anos, a presença desta área
nas pesquisas em áreas consideradas básicas como Física, Matemátia e Biologia de forma aplicada e também em um contexto multidisciplinar é cada vez mais
constante.

Este trabalho utilizou Ciência da Computação como um fim em um contexto de pesquisa que a utilizava como meio, demonstrando sua versatilidade. Nele pudemos
aplicar uma metodologia tradicional de Engenharia de Software com o desafio te prover um ambiente de fácil utilização e controle do processo pelos usuários,
e também fácil manutenção, utilizando-se de uma organização arquitetural em camadas, sem a perda das características originais do método de análise
filogenética através da integração com código legado, e incluindo a possibilidade de extensibilidade.

Como mostrado em capítulos anteriores, a área de Bioinformática vem crescendo bastante nos últimos anos, e sua demanda por pesquisa em Computação Aplicada
apresenta uma curva de crescimento exponencial, de modo que a concepção de um ambiente integrado para análise filogenética constitui de um trabalho de
desenvolvimento importante neste contexto.

O trabalho realizado aqui representa um ponto de partida para que não só um método possa ser executado, mas para que se tenha um ambiente muito maior, que
permita a aplicação de outros métodos de análise filogenética, como também a comparação e a extração de dados estatísticos entre eles. A modelagem do
ambiente permitirá a concepção de novas versões do protótipo, permitindo a continuação do modelo de desenvolvimento em espiral, e a modelagem pode
ser posteriormente ampliada, incluindo novos requisitos e funcionalidades e aumentando seu escopo. A próxima seção listará o que pode ser feito como trabalhos
futuros.

\section{Trabalhos futuros}

Os principais trabalhos futuros estão relacionados ao protótipo. Posteriormente, pode-se impliar o escopo e consequentemente a modelagem.Seguem os trabalhos
futuros:

\begin{itemize}
  \item{Melhorar o armazenamento e serialização de objetos: a persistência foi tratada como pacote na modelagem pela sua tendência de variação conforme
a plataforma. Na implementação foi criado um modelo simples de persistência, a partir da serialização dos objetos do sistema em um único arquivo. Ao
carregar um projeto, o sistema lê esse arquivo e recupera o estado original;}
  \item{Algumas operações exclusivas da camada de persistência ficaram na lógica de negócio. É necessário mover a adaptar os códigos onde isso ocorre;}
  \item{Execução remota de algoritmos: alguns métodos de clusterização levam alguns dias para executar. Muitas vezes é interessante executá-lo em uma outra
máquina de forma remota. É interessante haver uma interface ssh (Secure Shell) para a execução remota de algumas análises, à escolha do usuário;}
  \item{Uso de \textit{threads}: algumas análises podem ser realizadas com o uso de \textit{threads} para liberar a interface para outros usos;}
  \item{Alguns recursos do Qt podem ser melhor utilizados;}
  \item{A interface gráfica, apesar de representar uma melhoria em termos de uso, precisa ser melhorada;}
  \item{Quatro casos de uso não foram implementados no protótipo: definir módulos/comunidades (caso de uso 8), comparar/executar congruência de redes de
diferentes projetos (caso de uso 9), visualizar gráfico completo da rede (caso de uso 10), executar congruência em relação a um arquivo cadastrado em um
formato padrão (caso de uso 11) e exportar dados do projeto para gráficos em um formato de programas plotadores (caso de uso 12);}
  \item{A integração do sistema (feito em Python) com os módulos em Fortran está utilizando chamadas do sistema operacional. Python oferece um
\textit{binding} para Fortran que poderia ser utilizado;}
  \item{O protótipo não avisa ao usuário sobre algumas operações que estão ocorrendo e nem o progresso de algumas análises duradouras;}
\end{itemize}

\sigla{SSH}{Secure Shell}

\appendix

\chapter{Anexo I - Diagrama de Classes completo}

\begin{figure}
\centering
\includegraphics[scale=0.28]{diagrama-classes-completo}
\caption{Diagrama de classes completo do Navi.}
\label{fig:diagrama-classes-completo}
\end{figure}


\bibliography{monografia}

\input{glossario.tex}

\end{document}

