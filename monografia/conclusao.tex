\chapter{Conclusão} \label{cap:conclusao}

Este trabalho
se utilizou de práticas de Engenharia de Software com o desafio de prover um ambiente para o controle do processo pelos usuários,
e também oferecer aos desenvolvedores possibilidade de manutenção e extensibilidade, utilizando-se de uma organização arquitetural em camadas,
sem a perda das características originais do método de análise
filogenética através da integração com código legado, e incluindo a possibilidade de extensibilidade.

O primeiro protótipo do AIAF pode obter arquivos do GenBank, filtrá-los e a partir da escolha das sequências pelo usuário gerar a matriz de similaridades.
A partir dela, permite ao usuário a geração de matrizes de adjacência e vizinhança, e em seguida analisar limiares e executar a clusterização.
Além disso, possibilita a criação, carregamento e salvamento de projetos. O conjunto dessas funcionalidades
constitui na finalização do primeiro ciclo do modelo em espiral.

O trabalho realizado aqui representa um ponto de partida para que não só um método possa ser executado, mas para que se tenha um ambiente muito maior, que
possibilite a aplicação de outros métodos de análise filogenética, como também a comparação e a extração de dados estatísticos entre eles. A modelagem do
ambiente permitirá a concepção de novas versões do protótipo, levando à continuação do modelo de desenvolvimento em espiral, e a modelagem pode
ser posteriormente ampliada, incluindo novos requisitos e funcionalidades e aumentando seu escopo. A próxima seção listará o que pode ser feito como trabalhos
futuros.

\section{Trabalhos futuros}

Os principais trabalhos futuros estão relacionados ao protótipo. Posteriormente, pode-se ampliar o escopo e consequentemente a modelagem. Seguem os trabalhos
futuros:

\begin{itemize}
  \item{Dois casos de uso importantes que não foram implementados no protótipo são a definição de módulos/comunidades (caso de uso 7), onde o usuário define
com base na matriz de cores e no dengrograma quais sequências pertencem a quais comunidades e com isso pode ser gerado o gréfico completo da rede (caso de
uso 9) para visualização, onde o usuário pode ver as comunidades separadas por cores, além de obter facilmente informações sobre as sequências clicando nos
vértices. Estas funcionalidades em conjunto
exigem um trabalho de interface gráfica grande, incluindo a exibição interativa de gráficos e a integração de bibliotecas de exibição de
gráficos a bibliotecas de plotagem.}
  \item{Outros casos de uso também não foram implementados no protótipo, principalmente por questões de tempo: comparar/executar congruência de redes de
diferentes projetos (caso de uso 8), executar congruência em relação a um arquivo cadastrado em um
formato padrão (caso de uso 10) e exportar dados do projeto para gráficos em um formato de programas plotadores (caso de uso 11);}
  \item{A validação do protótipo é uma etapa muito importante neste processo de desenvolvimento e não foi realizada por conta de tempo e também pelo fato
de o mesmo não apresentar maturidade suficiente. Após a implementação de alguns casos de uso e soluções de questões de interface gráfica e comunicação com
o usuário como veremos abaixo, sua validação torna-se imprescindível;}
  \item{Fazer uma modelagem bem-definida do pacote de persistência;}
  \item{Melhorar o armazenamento e serialização de objetos: a persistência foi tratada como pacote na modelagem pela sua tendência de variação conforme
a plataforma. Na implementação foi criado um modelo simples de persistência, a partir da serialização dos objetos do sistema em um único arquivo. Ao
carregar um projeto, o sistema lê esse arquivo e recupera o estado original;}
  \item{O controle do pacote de persistência ficou centralizado na classe \textit{Project}. É necessário uma análise sobre as vantagens dentro
do projeto de tornar este controle distribuído;}
  \item{Execução remota de algoritmos: alguns métodos de clusterização levam alguns dias para executar. Muitas vezes é interessante executá-lo em uma outra
máquina de forma remota. É interessante haver uma interface ssh (Secure Shell) para a execução remota de algumas análises, à escolha do usuário;}
  \item{A interface gráfica, apesar de representar uma melhoria em termos de uso, precisa ser melhorada com a criação de \textit{widgets} em substituição
às janelas \textit{popup}, utilizando melhor os recursos do Qt;}
  \item{O protótipo não avisa ao usuário sobre algumas operações que estão ocorrendo e nem o progresso de algumas análises, o que é algo importante pois
determinadas análises demoram muito tempo para serem realizadas, tornando esse \textit{feedback} para o usuário algo essencial;}
  \item{Uso de \textit{threads}: algumas análises podem ser realizadas com o uso de \textit{threads} para liberar a interface gráfica, para o usuário possa
realizar outras operações no sistema enquanto alguma análise estiver em andamento;}
  \item{A integração do sistema (feito em Python) com os módulos em Fortran está utilizando chamadas do sistema operacional. Python oferece um
\textit{binding} para Fortran que poderia ser utilizado, pois proporcionaria a integração direta entre as duas linguagens sem utilizar chamadas ao
sistema operacional como uma espécie de camada intermediária para a comunicação entre elas.}
\end{itemize}

\sigla{SSH}{\textit{Secure Shell}}