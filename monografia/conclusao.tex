\chapter{Conclusão} \label{cap:conclusao}

Este trabalho
aplicou uma metodologia tradicional de Engenharia de Software com o desafio de prover um ambiente de fácil utilização e controle do processo pelos usuários,
e também fácil manutenção, utilizando-se de uma organização arquitetural em camadas, sem a perda das características originais do método de análise
filogenética através da integração com código legado, e incluindo a possibilidade de extensibilidade.

O primeiro protótipo do AIAF pode obter arquivos do GenBank, filtrá-los e a partir da escolha das sequências pelo usuário gerar a matriz de similaridades.
A partir dela, permite ao usuário a geração de matrizes de adjacência e vizinhança, e em seguida analisar limiares e executar a clusterização.
Além disso, possibilita a criação, carregamento e salvamento de projetos. O conjunto dessas funcionalidades
constitui na finalização do primeiro ciclo do modelo em espiral.

O trabalho realizado aqui representa um ponto de partida para que não só um método possa ser executado, mas para que se tenha um ambiente muito maior, que
possibilite a aplicação de outros métodos de análise filogenética, como também a comparação e a extração de dados estatísticos entre eles. A modelagem do
ambiente permitirá a concepção de novas versões do protótipo, levando à continuação do modelo de desenvolvimento em espiral, e a modelagem pode
ser posteriormente ampliada, incluindo novos requisitos e funcionalidades e aumentando seu escopo. A próxima seção listará o que pode ser feito como trabalhos
futuros.

\section{Trabalhos futuros}

Os principais trabalhos futuros estão relacionados ao protótipo. Posteriormente, pode-se ampliar o escopo e consequentemente a modelagem. Seguem os trabalhos
futuros:

\begin{itemize}
  \item{Cinco casos de uso não foram implementados no protótipo: definir módulos/comunidades (caso de uso 8), comparar/executar congruência de redes de
diferentes projetos (caso de uso 9), visualizar gráfico completo da rede (caso de uso 10), executar congruência em relação a um arquivo cadastrado em um
formato padrão (caso de uso 11) e exportar dados do projeto para gráficos em um formato de programas plotadores (caso de uso 12);}
  \item{Fazer uma modelagem bem-definida do pacote de persistência;}
  \item{Melhorar o armazenamento e serialização de objetos: a persistência foi tratada como pacote na modelagem pela sua tendência de variação conforme
a plataforma. Na implementação foi criado um modelo simples de persistência, a partir da serialização dos objetos do sistema em um único arquivo. Ao
carregar um projeto, o sistema lê esse arquivo e recupera o estado original;}
  \item{Algumas operações exclusivas da camada de persistência ficaram na lógica de negócio. É necessário mover a adaptar os códigos onde isso ocorre;}
  \item{Execução remota de algoritmos: alguns métodos de clusterização levam alguns dias para executar. Muitas vezes é interessante executá-lo em uma outra
máquina de forma remota. É interessante haver uma interface ssh (Secure Shell) para a execução remota de algumas análises, à escolha do usuário;}
  \item{A interface gráfica, apesar de representar uma melhoria em termos de uso, precisa ser melhorada;}
  \item{Alguns recursos do Qt podem ser melhor utilizados;}
  \item{O protótipo não avisa ao usuário sobre algumas operações que estão ocorrendo e nem o progresso de algumas análises duradouras;}
  \item{Uso de \textit{threads}: algumas análises podem ser realizadas com o uso de \textit{threads} para liberar a interface para outros usos;}
  \item{A integração do sistema (feito em Python) com os módulos em Fortran está utilizando chamadas do sistema operacional. Python oferece um
\textit{binding} para Fortran que poderia ser utilizado.}
\end{itemize}

\sigla{SSH}{\textit{Secure Shell}}